\begin{titlepage}
  \begin{center}

  {\Huge CLOCK\_STIMULATOR}

  \vspace{25mm}

  \includegraphics[width=0.90\textwidth,height=\textheight,keepaspectratio]{img/AFRL.png}

  \vspace{25mm}

  \today

  \vspace{15mm}

  {\Large Jay Convertino}

  \end{center}
\end{titlepage}

\tableofcontents

\newpage

\section{Usage}

\subsection{Introduction}

\par
This modules creates multiple clocks and multiple negative or positive resets.
The module outputs a vector based on the number requested.

\subsection{Dependencies}

\par
The following are the dependencies of the cores.

\begin{itemize}
  \item fusesoc 2.X
  \item iverilog (simulation)
  \item cocotb (simulation)
\end{itemize}

% \input{src/fusesoc/depend_fusesoc_info.tex}

\section{Architecture}
\par
Generates a clock and reset for simulation. The clock is for loop generated using delay controls in a always block.
The reset is geneated in a for loop for each reset in a initial block with a delay based upon the increment amount.

Please see \ref{Module Documentation} for more information per target.

\section{Building}

\par
The all clock stimulator modules are written in Verilog 2001. They should synthesize in any modern FPGA software. The core comes as a fusesoc packaged core and can be
included in any other core. Be sure to make sure you have meet the dependencies listed in the previous section.

\subsection{fusesoc}
\par
Fusesoc is a system for building FPGA software without relying on the internal project management of the tool. Avoiding vendor lock in to Vivado or Quartus.
These cores, when included in a project, can be easily integrated and targets created based upon the end developer needs. The core by itself is not a part of
a system and should be integrated into a fusesoc based system. Simulations are setup to use fusesoc and are a part of its targets.

\subsection{Source Files}

\subsubsection{fusesoc\_info File List}
\begin{itemize}
\item src
	\begin{itemize}
	\item src/tm\_stim\_clk.v
	\end{itemize}
\item tb
	\begin{itemize}
	\item tb/tb\_clk.v
	\end{itemize}
\end{itemize}


\subsection{Targets} \label{targets}

\subsubsection{fusesoc\_info Targets}
\begin{itemize}
\item default
	\begin{itemize}
	\item[$\space$] Info: Default file set.
	\end{itemize}
\item sim
	\begin{itemize}
	\item[$\space$] Info: Default for sim intergration.
	\end{itemize}
\end{itemize}


\subsection{Directory Guide}

\par
Below highlights important folders from the root of the directory.

\begin{enumerate}
  \item \textbf{docs} Contains all documentation related to this project.
    \begin{itemize}
      \item \textbf{manual} Contains user manual and github page that are generated from the latex sources.
    \end{itemize}
  \item \textbf{src} Contains source files for clock\_stimulator.
  \item \textbf{tb} Contains test bench files.
\end{enumerate}

\newpage

\section{Simulation}
\par
There is no simulation at the moment. Maybe a future addition?

\newpage

\section{Module Documentation} \label{Module Documentation}

\par
There project has multiple modules. The targets are the top system wrappers.

\begin{itemize}
\item \textbf{tm\_stim\_clk}
\item \textbf{tb\_clk}
\end{itemize}
The next sections document the module in great detail.

